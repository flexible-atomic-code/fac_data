\documentclass{revtex4}

\begin{document}

\author {M. F. Gu}

\affiliation{Prism Computational Sciences, WI, USA \\ Space Science Laboratory, UC Berkeley, CA, USA}

\title{Tutorial: FAC for Intermediate Users}

\maketitle

In this tutorial, we will discuss pratical usage of the Flexible Atomic Code (FAC) for intermediate users. Basic knowledge of atomic physics and plasma spectroscopy is assumed, but familiarity with FAC in particular is not required. The topics covered in the tutorial include:

  \begin{itemize}
  \item Download and installation.
  \item Atomic processes implemented in FAC.
  \item SFAC and PFAC interfaces.
  \item Handling FAC binary and ascii output files.
  \item Basic usage for simple atomic systems.
  \item Coupling atomic calculations and collisional radiative modeling.
  \item Advanced features: Many-body perturbation theory and R-Matrix.
  \end{itemize}

  We will follow the tutorial through several practical examples, both simple and more advanced. Simple examples will demonstrate the use of FAC in obtaining limited sets of atomic parameters, such as energy levels, radiative rates, and collisional cross sections. More complex examples will demonstate the use of PFAC interface (via the Python scripting language) to generate large scale datasets, and use them as input to collisional radiative models for spectroscopic modeling applications.
  
\end{document}
